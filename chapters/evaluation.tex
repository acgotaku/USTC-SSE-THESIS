\chapter{系统测试与分析}

在实现完之后,要通过详尽的测试来保证系统的健全。

\section{测试平台的搭建}

本文对烟雾运动模拟系统进行了详细的测试,测试的环境为:

\begin{itemize}
  \item 硬件平台:Intel Core i7-4790处理器,32GB内存,NVIDIA GeForce GTX 860独立显卡,2GB显存。
  \item 软件平台:Windows 10 64bit操作系统,Chrome 55 浏览器,Sublime Text文本编辑器。
\end{itemize}


\section{测试的目的}

对软件系统的测试,是软件开发周期的最后一道工序也是软件质量的保证。我们应该在测试的过程中尽量发现软件的缺陷并进行修补。在不同的开发阶段修补缺陷的成本是不一样的,软件BUG发现的越晚,特别是在软件发布之后,我们为修补BUG所付出的代价就越大,有时还会影响用户体验。因此对软件进行测试是保证软件质量的有效手段,我们应当在发布系统之前进行详尽的测试,测试达到的主要目标有:

\begin{enumerate}
\item 系统的功能性
\item 系统的运行性能
\item 系统的稳定性
\item 系统的视觉效果
\end{enumerate}


\section{系统的功能性测试}

针对需求分析里提出的功能点,逐一进行测试。

\section{系统的稳定性测试}

测试系统在各种边界条件下的稳定性,保证运行不会崩溃或有内存泄漏等意外情况发生。

\section{系统的运行性能测试}


针对系统的运行性能和效率进行测试,可以采用各种测试工具来形象的说明测试结果。


\section{系统的视觉效果测试}

这是针对图形学论文的特殊测试,其他项目可以不需要这个子标题。

\section{本章小结}

除了绪论,每章的最后一节都必须有本章小结,这是学校的规定。