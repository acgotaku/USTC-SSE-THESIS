\begin{abstract}
衰老使得正常的视觉功能受到严重影响,眼睛光学系统的老年性改变并不足以解释这种视觉功能衰退,一般认为是神经系统的退化导致了这种老年性功能降低。过去几年中在老年动物视觉皮层发现了一系列细胞反应特性的改变,这些细胞水平的变化被认为是老年性视觉功能衰退的神经机制。为了更全面的了解衰老过程对视觉皮层的影响以及细胞反应改变与整体功能降低之间的关系,本研究采用在体细胞单位外单位记录方法,从以下三个方面进一步比较了青年和老年猕猴视觉皮层细胞反应特性的差异:

以前的研究发现在老年猕猴初级视皮层(V1),细胞正常的反应特性发生了与功能退化相关的改变。为了研究这种细胞反应特性的改变如何在皮层视觉通路中传递,本实验研究了衰老对猕猴次级视皮层(V2)细胞功能的影响。我们发现,与年轻猴相比,老年猕猴V2细胞的方位、方向选择性明显下降,同时自发放电活动增加,视觉刺激诱发的反应幅度增加以及细胞活动的信噪比降低。由于V2区在复杂视觉信息处理过程中有重要作用,本结果提供了衰老影响较高级视觉功能可能的神经机制。与V1区细胞的老年性功能退化相比较,V2区细胞反应特性受老化的影响更为严重。与V1区细胞的老年性功能退化相比较,V2区细胞反应特性受老化的影响更为严重,这说明V2区表现出的功能退化并非完全由V1区的变化引起,衰老的影响可能在皮层视觉通路中逐渐积累。

通过上述三个实验,我们研究了灵长类动物视皮层细胞老年性功能退化在不同脑区的表现特点,随衰老而变化的动态过程以及细胞信息处理能力所受衰老的影响。这些结果更加完整的描述了老年性视觉功能下降的神经机制。

论文摘要可以以列表的形式列出自己的主要工作:
\begin{enumerate}
\item 本论文提出了XXX的解决方案。
\item 本论文使用XXX技术进行实现。
\item 本论文基于XXX。
\end{enumerate}

\keywords{衰老,功能退化,初级视皮层,次级视皮层,兴奋-抑制平衡,信息处理,猕猴}
\end{abstract}

\begin{enabstract}
Visual function is influenced significantly by aging. The aging-related visual deficits can be observed even after the optical factor has been well controlled, suggesting the important role the central nerve system played in generation of such deficits. In the past few years, several alterations in response properties of visual cortical cells have been reported in senescent animal models, which were considered as the neural mechanisms underlying the aging-related degradation of vision. In order to get a more comprehensive understanding about the aging effects on visual cortices and the relationship between the changes at cellular level and the functional degeneration at the whole system level, in the present study, we used in vivo extracellular single-unit recording techniques to compare the response property of visual cortical eclls in young and old monkeys. The study included three experiments:

By these three experiments, we studied the different characteristics of aging effects in different brain areas, the dynamics of these effects and the impairments of signal processing during senescence. All of these findings provided us a more comprehensive understanding about the mechanisms underlying the aging-related degeneration of visual function.


\enkeywords{aging, function degeneration, primary visual, secondary visual cortex, balance between excitation and inhibition, signal processing, rhesus monkey}
\end{enabstract}
